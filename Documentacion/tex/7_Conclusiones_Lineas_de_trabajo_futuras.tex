\capitulo{7}{Conclusiones y Líneas de trabajo futuras}

En este apartado se exponen las conclusiones obtenidas tras finalizar el proyecto, además de explicar posibles líneas de mejora futuras para el mismo.

\subsection{Conclusiones}
Tras ocho meses de trabajando en el desarrollo, además de trabajar fuera de la Universidad, considero que he cumplido los objetivos del proyecto, y que la aplicación resultante de todo el trabajo puede llegar a ser realmente útil, no solo para fomentar los desplazamientos a pie en la propia ciudad, si no que también a la hora de viajar y conocer nuevas ciudades.


Destacar la gran cantidad de conocimientos nuevos aprendidos, no solo en cuanto a programación, si no también en cuando a herramientas y capacidad de reunir información. He intentado usar funcionalidades y herramientas nuevas, como los frameworks y librerías explicados en el apartado \nameref{herramientas}. Creo que el resultado final es funcional, útil, y como mencionaba, me ha permitido adquirir nuevos conocimientos.


Por último, después de varios años en la universidad, ser capaz de desarrollar un proyecto desde su inicio y que el resultado final sea una aplicación entregabley funcional es un gran logro, y personalmente, estoy muy contento de ello.


\subsection{Líneas de trabajo futuras}\label{futuras}
Como futuras evoluciones del proyecto, se plantean las siguientes ideas:

\begin{itemize}
	\item Mejorar la seguridad de la aplicación.
	\item Mejorar la tolerancia a fallos.
	\item Añadir una base de datos de tal forma que cada usuario pueda guardar y editar sus metrominutos.
	\item Mejorar la apariencia y maquetación de la aplicación.
	\item Añadir decoración al mapa sinópico como estaciones de transporte público, parques, ríos y costa.
	\item Añadir la localización, de tal modo que el usuario pueda identificar fácilmente en que punto del recorrido se encuentra.
\end{itemize}
