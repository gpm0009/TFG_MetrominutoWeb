\capitulo{7}{Conclusiones y Líneas de trabajo futuras}

En este apartado se exponen las conclusiones obtenidas tras finalizar el proyecto, además de explicar posibles líneas de mejora futuras para el mismo.

\subsection{Conclusiones}
Durante los ocho o nueve meses que he estado desarrollando este proyecto, también he estado trabajando fuera de la Universidad. Durante este tiempo, como se puede observar en el repositorio del proyecto\footnote{\url{https://github.com/gpm0009/TFG_MetrominutoWeb}}, el trabajo ha sido constante, y como consecuencia de ello, considero que la aplicación final de todo el trabajo puede llegar a ser realmente útil, no solo para fomentar los desplazamientos a pie en la propia ciudad, si no que también a la hora de viajar y conocer nuevas ciudades.


Hay que destacar también la gran cantidad de conocimientos nuevos aprendidos, no solo en cuanto a programación, sino también en cuando a diseño, teoría de grafos, uso de APIs, uso de plataformas Cloud... y de la capacidad para reunir información o buscar documentación. He intentado usar funcionalidades y herramientas nuevas, como los \emph{frameworks} y bibliotecas explicados en el apartado \nameref{herramientas}.


Finalmente, a nivel personal, me genera una gran satisfacción haber sido capaz de desarrollar, desde su planificación e inicio, hasta su finalización, una aplicación entregable, funcional y sobretodo, útil.


\subsection{Líneas de trabajo futuras}\label{futuras}
Como futuras evoluciones del proyecto, se plantean las siguientes ideas:

\begin{itemize}
	\item Mejorar la seguridad de la aplicación: ocultar en la consola del navegador las claves mediante algún sistema de seguridad.
	\item Mejorar la tolerancia a fallos: mejorar la respuesta de la aplicación ante fallos o errores.
	\item Añadir una base de datos: con esta mejora podríamos conseguir personalizar la aplicación para cada usuario, de tal modo que este pueda guardar y editar en cualquier momento sus metrominutos.
	\item Mejorar la apariencia y maquetación de la aplicación: mejorar los estilos de la aplicación.
	\item Añadir decoración al mapa sinópico como estaciones de transporte público, parques, lagos, ríos o zonas de costa, en el caso de que existan.
	\item Añadir geolocalización (provista por otro API de Google, Geolocation\footnote{\url{https://developers-dot-devsite-v2-prod.appspot.com/maps/documentation/geolocation/intro}}): de este modo el usuario puede identificar fácilmente en qué punto del recorrido se encuentra.
	\item Permitir al usuario en el SVG agrupar zonas, dibujar parques o añadir elementos que considere interesantes.
	\item Tras la modificación de la posición de textos o puntos por parte del usuario, dar la opción al usuario de recuperar el <<magnetismo>> entre textos y nodos manteniendo la posición actual de los mismo, de tal modo que si vuelve a mover el punto, los textos se muevan junto a él.
	\item Añadir a la aplicación traducciones, de manera que los textos aparezcan en el idioma que escoja el usuario.
\end{itemize}
