\capitulo{2}{Objetivos del proyecto}

A continuación, se detallan los diferentes objetivos que han motivado la
realización del proyecto.

\subsection{Objetivos Principales}

\begin{itemize}
	\item Desarrollar una aplicación web en la que los diferentes usuarios puedan seleccionar diversos puntos en un mapa (ciudad) con el fin de recorrerlos de la forma más óptima posible.
	\item Generar un grafo de nodos a partir de los puntos seleccionados en el mapa.
	\item Visualizar los puntos seleccionados: información acerca de la ubicación, marcador en el mapa.
	\item Permitir al usuario añadir y eliminar líneas (conexiones) entre puntos.
	\item Calcular el trayecto mas corto para el usuario de manera que pase por todos los puntos seleccionados.
\end{itemize}

\subsection{Objetivos Técnicos}
\begin{itemize}
	\item Desarrollar una aplicación cliente servidor en Python utilizando Flask.
	\item Hacer uso del API de Google para obtener la localización de puntos de interés sobre sus mapas.
	\item Hacer uso de Git como sistema de control de versiones del proyecto.
	\item Aplicar la teoría de grafos.
	\item Usar redes para el cálculo de los grafos.
\end{itemize}
\subsection{Objetivos a nivel personal}
\begin{itemize}
	\item Realizar una aportación al turismo derivada de una necesidad personal.
	\item Poner en práctica los conocimientos adquiridos durante el Grado para el correcto desarrollo del Trabajo de Fin de Grado.
\end{itemize}
