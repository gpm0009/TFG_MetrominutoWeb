\capitulo{1}{Introducción}
-- Movilidad Urbana.

Hoy en día la movilidad en los centros urbanos es mayoritariamente motorizada y ocupa al rededor del 70\% del espacio público.


-- Metrominutos, lo que son y dónde se aplican

Ante este problema de infravaloración de los desplazamientos a pie surge la idea de  Metrominuto~\cite{metrominuto}, que consiste en un mapa sinóptico que une diferentes puntos de la ciudad en función de la distancia existente entre cada uno de ellos. Su propósito es animar a los ciudadanos a moverse por la ciudad, lo cual supone beneficios en muchos aspectos: tanto de salud, como de contaminaciones.

Actualmente ya existen ciudades con Metrominutos como Pontevedra (pionera en esta idea), Sevilla, Madrid o León, pero este proyecto lo que trata es de automatizar este proceso de creación de mapas de manera que es el propio usuario quien selecciona los puntos que van a aparecer en él.



-- Lo que va a hacer el proyecto: ayudar a hacer metrominutos automáticos

El propósito de este proyecto es la creación automática de estos metrominutos explicados anteriormente. De este modo, cada usuario podrá tener a su disposición su propio Metrominuto personalizado, eliminando puntos que para dicho usuario no sean de interés y añadiendo los que si lo son.
