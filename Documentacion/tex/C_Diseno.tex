\apendice{Especificación de diseño}

\section{Introducción}
En este apartado de la documentación se expone el diseño que ha dado lugar a la aplicación, el cual incluye el diseño de las distintas estructuras de datos, el diseño procedimental y diseño arquitectónico.
\section{Diseño de datos}
La estructura del proyecto podemos dividirla en varias partes ya que, debido al uso de bibliotecas como la de Google Maps o de la biblioteca de Networkx para la generación de grafos, se han usado todos o parte de los datos que nos devuelven como resultado de las consultas.

\subsection{Networkx}
Como ya he mencionado, esta biblioteca nos permite trabajar de una forma muy amplia y completa con grafos. A lo largo del proyecto se han usado:

//explico la biblioteca o es buena idea poner función por función los parametros, lo que hace, y el resultado?


\subsection{Google API}
Google se ha usado para la obtención de todos los datos necesarios para el cálculo de distancias y tiempos, así como para la selección de los diferentes puntos en el mapa. Para ello, Google proporciona un API para \textit{Python} y otro para \textit{JavaScript}. Las funciones que se han usado, junto con los parámetro s de entrada y de salida han sido:

//Misma duda que arriba 


\section{Diseño procedimental}

//diagrama de secuencia entre servidor - apis - cliente ?

\section{Diseño arquitectónico}


