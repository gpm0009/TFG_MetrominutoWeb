\capitulo{6}{Trabajos relacionados}

En este apartado se explicarán y comentarán brevemente algunas aplicaciones de la idea original de Metrominuto, en las que la finalidad de todas ellas es el fomento del <<arte de caminar>>~\cite{wiki:metrominuto-pontevedra}.



\subsection{Metrominuto Pontevedra}

Pionera en esta idea, la ciudad de Pontevedra fue la primera en publicar un metrominuto (ver figura~\ref{fig:pontevedrapdf}). Hizo de caminar, un producto, convirtuendolo en la mejor alternativa para evitar el uso de vehículos motorizados.


Además, cuenta con una cuenta de facebook \url{https://www.facebook.com/metrominuto/} en la que añaden distintas publicaciones relacionadas con su producto, o simplemente publicaciones de metrominutos de otras ciudades, a las que se ha extendido la idea, que según el el artículo \textit{Al menos 57 ciudades han copiado el Metrominuto pontevedrés}~\cite{art:metromin-ciudades}, ya son, como bien indica el título, más de cincuenta y siete ciudades las que ya tienen uno.

\imagen{pontevedrapdf}{Metrominuto de Pontevedra.}


\subsection{Aplicaciones similares}

Conforme iba pasando el tiempo, no solo copiaron la idea de metrominuto aplicándola al ámbito urbano, sino que también trasladándola a otros ámbitos. Es por esto que surgen una serie de aplicaciones o herramientas similares que veremos a continuación.


\subsubsection{Pasominuto}
Pasominuto: \url{https://pontevedraviva.com/web/uploads/arquivos/pasominuto.pdf}.
Esta alternativa a Metrominuto consiste en una guía de recorridos para pasear por <<los espacios más agradables>> de la ciudad, teniendo en cuánta sus distancias, pasos y tiempos. El artículo~\cite{art:pasominuto} menciona hasta un total de 29 recorridos diferentes.
\imagen{pasominutoImagen}{Pasominuto.}


\subsubsection{Metroplayas}
Aplicación considera como <<hijo de metrominuto>> y que sitúa a Pontevedra en el punto equidistante de las mejores playas de las \textit{Rías Baixas}.
\imagen{metroplayaImagen}{Metroplaya.}


\subsection{Metro Pie}
Metro Pie consiste en un conjunto de trayectos predefinidos en los que se especifica el tiempo que se tarda en recorrerlos. ¿Su objetivo?. El mismo que antes, fomentar los desplazamientos a pie. Como ejemplo, una de las ciudades que ha incorporado esta idea es Santander, como nos muestra el artículo \textit{Santander lanza el Metropie: un mapa con itinerarios por la ciudad para fomentar los desplazamientos peatonales}~\cite{art:metropie}.
\imagen{metropie-santander}{Metro Pie de Santander.}
