\apendice{Especificación de Requisitos}

\section{Introducción}
En este apéndice se explican y especifican tanto los requisitos funcionales como los no funcionales del proyecto, así como los objetivos del proyecto.

\section{Objetivos generales}

\begin{itemize}
	\item Crear una aplicación cliente -- servidor que permita la creación automática de Metrominutos.
    \item Control de usuarios.
    \item Permitir a los usuarios control sobre el mapa, de manera que puedan mover o eliminar los puntos seleccionados.
    \item Ofrecer al usuario un mapa final claro y sencillo.
    \item Que la aplicación final sea útil para el fomento de esta actividad.
\end{itemize}

\section{Catálogo de requisitos}

\subsection{Requisitos funcionales}
\begin{itemize}
	\item \textbf{RF-1 Control de usuarios}: la aplicación debe permitir el control de usuarios.
	\begin{itemize}
		\item \textbf{RF-1.1 Firebase Auth:} La aplicación debe poder hacer uso de los servicios de autenticación de Firebase.
	\end{itemize}
	\item \textbf{RF-2 Generación de mapa:} La aplicación debe ofrecer la selección de distintos puntos sobre el mapa para generar un mapa personalizado.
	\begin{itemize}
		\item \textbf{RF-2.1 Visualización de puntos sobre Google Maps:} La aplicación debe poder hacer uso de las operaciones, así como acceder a los servicios de \textit{Google Maps} proporcionados por el API.
		\item \textbf{RF-2.2 Selección de puntos:} Seleccionar sobre el mapa distintos puntos, hasta un máximo de 15~\footnote{limitación que se debe al número de argumentos que permite el API de Distance Matrix de Google.}.
		\begin{itemize}
			\item \textbf{RF-2.2.1 Selección como centrales:} Una vez creado un marcador, el usuario debe poder darle una mayor importancia a este punto en el recorrido.
			\item \textbf{RF-2.2.2 Eliminación de un punto cualquiera:} El usuario debe poder eliminar los puntos seleccionados. 
		\end{itemize}
		
	\end{itemize}
	\item \textbf{RF-3 Creación de SVG:} Generación de un mapa sinóptico a través de los puntos seleccionados en el \textit{RF-~2}, con formato SVG, que permite la interacción con dicho mapa.
	\begin{itemize}
		\item \textbf{RF-3.1 Añadir y eliminar arcos:} El usuario debe poder seleccionar el mapa o grafo que más se adecue a su necesidad, con más o menos recorridos (arcos).
		\item \textbf{RF-3.2 Mover puntos y etiquetas:} Posibilidad de que el usuario recoloque tanto los puntos como las etiquetas de los mismos en el mapa.
		\item \textbf{RF-3.3 Cambiar el nombre de los puntos:} El usuario debe poder cambiar las etiquetas referentes al nombre de los puntos.
		\item \textbf{RF-3.4 Exportación y descarga:} El usuario debe poder descargar el mapa que ha generado.
	\end{itemize}
\end{itemize}


\subsection{Requisitos no funcionales}
\begin{itemize}
	\item \textbf{RNF-1 Usabilidad:} La aplicación tiene que poder usarse de forma sencilla y debe ser intuitiva.
	\item \textbf{RNF-2 Compatibilidad:} La aplicación tiene que poder ser compatible con los diferentes navegadores.
	\item \textbf{RNF-3 Responsividad:} La aplicación debe poder adaptarse al tamaño de la pantalla.
	\item \textbf{RNF-4 Facilidad en el despliegue:} La aplicación debe poder ser desplegada con facilidad en un servidor.
	\item \textbf{RFN-5 Mantenibilidad:} Debe ser sencillo añadir nuevas funcionalidades.
\end{itemize}

\section{Especificación de requisitos}
En esta sección, se presentan los casos de uso de la aplicación.
\imagen{Diagrama de caso de uso - sitema}{Diagrama de casos de uso: General.}
\imagen{Diagrama de caso de uso - RF2}{Diagrama de casos de uso: RF2 -- Generación de mapa.}
\imagen{Diagrama de caso de uso - RF3}{Diagrama de casos de uso: RF3 -- Creación de SVG.}


%Caso de Uso 1%
\begin{longtable}[H]{@{}l|l@{}}
	\toprule
	\begin{minipage}[b]{0.23\columnwidth}\raggedright\strut
		\textbf{CU-1}\strut
	\end{minipage} & \begin{minipage}[b]{0.71\columnwidth}\raggedright\strut
		\textbf{Iniciar sesión}\strut
	\end{minipage}\tabularnewline
	\toprule
	\endhead
	\begin{minipage}[t]{0.23\columnwidth}\raggedright\strut
		\textbf{Requisitos asociados}\strut
	\end{minipage} & \begin{minipage}[t]{0.71\columnwidth}\raggedright\strut
		RF-1, RF-1.1\strut
	\end{minipage}\tabularnewline
	\midrule
	\begin{minipage}[t]{0.23\columnwidth}\raggedright\strut
		\textbf{Actor}\strut
	\end{minipage} & \begin{minipage}[t]{0.71\columnwidth}\raggedright\strut
		Usuario no logueado.\strut
	\end{minipage}\tabularnewline
	\midrule
	\begin{minipage}[t]{0.23\columnwidth}\raggedright\strut
		\textbf{Descripción}\strut
	\end{minipage} & \begin{minipage}[t]{0.71\columnwidth}\raggedright\strut
		El usuario inicia sesión en la aplicación.\strut
	\end{minipage}\tabularnewline
	\midrule
	\begin{minipage}[t]{0.23\columnwidth}\raggedright\strut
		\textbf{Precondición}\strut
	\end{minipage} & \begin{minipage}[t]{0.71\columnwidth}\raggedright
		\begin{itemize}
			\tightlist
			\item Página de inicio cargada.
			\item Usuario no logueado.
		\end{itemize}
	\end{minipage}\tabularnewline
	\midrule
	\begin{minipage}[t]{0.23\columnwidth}\raggedright\strut
		\textbf{Acciones}\strut
	\end{minipage} & \begin{minipage}[t]{0.71\columnwidth}\raggedright
		\begin{enumerate}
			\def\labelenumi{\arabic{enumi}.}
			\item Pulsar sobre el botón Iniciar sesión.
		\end{enumerate}
	\end{minipage}\tabularnewline
	\midrule
	\begin{minipage}[t]{0.23\columnwidth}\raggedright\strut
		\textbf{Postcondición}\strut
	\end{minipage} & \begin{minipage}[t]{0.71\columnwidth}\raggedright\strut
		Redirección a la aplicación con la sesión iniciada.\strut
	\end{minipage}\tabularnewline
	\midrule
	\begin{minipage}[t]{0.23\columnwidth}\raggedright\strut
		\textbf{Excepciones}\strut
	\end{minipage} & \begin{minipage}[t]{0.71\columnwidth}\raggedright
		\begin{itemize}
			\tightlist
			\item Cuenta no existente.
			\item Combinación de nombre y contraseña incorrecta.
			\item Acceso no permitido: usuario bloqueado.
		\end{itemize}
	\end{minipage}\tabularnewline
	\midrule
	\begin{minipage}[t]{0.23\columnwidth}\raggedright\strut
		\textbf{Importancia}\strut
	\end{minipage} & \begin{minipage}[t]{0.71\columnwidth}\raggedright\strut
		Alta\strut
	\end{minipage}\tabularnewline
	\bottomrule
	\caption{Iniciar sesión.}
	\label{cu:1}
\end{longtable}
\newpage


%Caso de Uso 1.1%
\begin{longtable}[H]{@{}l|l@{}}
	\toprule
	\begin{minipage}[b]{0.23\columnwidth}\raggedright\strut
		\textbf{CU-1.1}\strut
	\end{minipage} & \begin{minipage}[b]{0.71\columnwidth}\raggedright\strut
		\textbf{Firebase Auth}\strut
	\end{minipage}\tabularnewline
	\toprule
	\endhead
	\begin{minipage}[t]{0.23\columnwidth}\raggedright\strut
		\textbf{Requisitos asociados}\strut
	\end{minipage} & \begin{minipage}[t]{0.71\columnwidth}\raggedright\strut
		RF-1.1\strut
	\end{minipage}\tabularnewline
	\midrule
	\begin{minipage}[t]{0.23\columnwidth}\raggedright\strut
		\textbf{Actor}\strut
	\end{minipage} & \begin{minipage}[t]{0.71\columnwidth}\raggedright\strut
		Usuario no logueado.\strut
	\end{minipage}\tabularnewline
	\midrule
	\begin{minipage}[t]{0.23\columnwidth}\raggedright\strut
		\textbf{Descripción}\strut
	\end{minipage} & \begin{minipage}[t]{0.71\columnwidth}\raggedright\strut
		El usuario inicia sesión mediante el proveedor que seleccione.\strut
	\end{minipage}\tabularnewline
	\midrule
	\begin{minipage}[t]{0.23\columnwidth}\raggedright\strut
		\textbf{Precondición}\strut
	\end{minipage} & \begin{minipage}[t]{0.71\columnwidth}\raggedright\strut
		\begin{itemize}
			\item Dominio autorizado en la consola de Firebase por el administrador.
			\item Proveedor correctamente habilitado por el administrador.
		\end{itemize}
	\end{minipage}\tabularnewline
	\midrule
	\begin{minipage}[t]{0.23\columnwidth}\raggedright\strut
		\textbf{Acciones}\strut
	\end{minipage} & \begin{minipage}[t]{0.71\columnwidth}\raggedright
		\begin{enumerate}
			\def\labelenumi{\arabic{enumi}.}
			\item Pulsar sobre el proveedor deseado.
			\item Introducir credenciales.
		\end{enumerate}
	\end{minipage}\tabularnewline
	\midrule
	\begin{minipage}[t]{0.23\columnwidth}\raggedright\strut
		\textbf{Postcondición}\strut
	\end{minipage} & \begin{minipage}[t]{0.71\columnwidth}\raggedright\strut
		Redirección a la aplicación con la sesión iniciada.\strut
	\end{minipage}\tabularnewline
	\midrule
	\begin{minipage}[t]{0.23\columnwidth}\raggedright\strut
		\textbf{Excepciones}\strut
	\end{minipage} & \begin{minipage}[t]{0.71\columnwidth}\raggedright
		\begin{itemize}
			\tightlist
			\item Cuenta no existente.
			\item Combinación de nombre y contraseña incorrecta.
			\item Acceso no permitido: usuario bloqueado.
		\end{itemize}
	\end{minipage}\tabularnewline
	\midrule
	\begin{minipage}[t]{0.23\columnwidth}\raggedright\strut
		\textbf{Importancia}\strut
	\end{minipage} & \begin{minipage}[t]{0.71\columnwidth}\raggedright\strut
		Alta\strut
	\end{minipage}\tabularnewline
	\bottomrule
	\caption{Firebase Auth.}
	\label{cu:1.1}
\end{longtable}
\newpage


%Caso de Uso 2%
\begin{longtable}[H]{@{}l|l@{}}
	\toprule
	\begin{minipage}[b]{0.23\columnwidth}\raggedright\strut
		\textbf{CU-2}\strut
	\end{minipage} & \begin{minipage}[b]{0.71\columnwidth}\raggedright\strut
		\textbf{Generación de mapa}\strut
	\end{minipage}\tabularnewline
	\toprule
	\endhead
	\begin{minipage}[t]{0.23\columnwidth}\raggedright\strut
		\textbf{Requisitos asociados}\strut
	\end{minipage} & \begin{minipage}[t]{0.71\columnwidth}\raggedright\strut
		RF-2, RF-2.1, RF-2.2, RF-2.1.1, RF-2.2.1, RF-2.2.2\strut
	\end{minipage}\tabularnewline
	\midrule
	\begin{minipage}[t]{0.23\columnwidth}\raggedright\strut
		\textbf{Actor}\strut
	\end{minipage} & \begin{minipage}[t]{0.71\columnwidth}\raggedright\strut
		Usuario logueado.\strut
	\end{minipage}\tabularnewline
	\midrule
	\begin{minipage}[t]{0.23\columnwidth}\raggedright\strut
		\textbf{Descripción}\strut
	\end{minipage} & \begin{minipage}[t]{0.71\columnwidth}\raggedright\strut
		El usuario selecciona distintos puntos sobre Google Maps y genera un mapa sinóptico.\strut
	\end{minipage}\tabularnewline
	\midrule
	\begin{minipage}[t]{0.23\columnwidth}\raggedright\strut
		\textbf{Precondición}\strut
	\end{minipage} & \begin{minipage}[t]{0.71\columnwidth}\raggedright\strut
		Sesión iniciada.\strut
	\end{minipage}\tabularnewline
	\midrule
	\begin{minipage}[t]{0.23\columnwidth}\raggedright\strut
		\textbf{Acciones}\strut
	\end{minipage} & \begin{minipage}[t]{0.71\columnwidth}\raggedright
		\begin{enumerate}
			\def\labelenumi{\arabic{enumi}.}
			\tightlist
			\item Pulsar sobre el mapa para marcar un punto.
			\item Pulsar sobre el icono de la papelera para borrar un punto.
			\item Pulsar sobre Borrar Marcadores para eliminar todos los puntos.
			\item Pulsar sobre Generar mapa para crear un mapa sinóptico de los puntos seleccionados.
		\end{enumerate}
	\end{minipage}\tabularnewline
	\midrule
	\begin{minipage}[t]{0.23\columnwidth}\raggedright\strut
		\textbf{Postcondición}\strut
	\end{minipage} & \begin{minipage}[t]{0.71\columnwidth}\raggedright\strut
		Redirección a la vista del mapa sinóptico.\strut
	\end{minipage}\tabularnewline
	\midrule
	\begin{minipage}[t]{0.23\columnwidth}\raggedright\strut
		\textbf{Excepciones}\strut
	\end{minipage} & \begin{minipage}[t]{0.71\columnwidth}\raggedright
		\begin{itemize}
			\tightlist
			\item Seleccionar más de 15 puntos.
			\item Seleccionar menos de 3 puntos.
		\end{itemize}
	\end{minipage}\tabularnewline
	\midrule
	\begin{minipage}[t]{0.23\columnwidth}\raggedright\strut
		\textbf{Importancia}\strut
	\end{minipage} & \begin{minipage}[t]{0.71\columnwidth}\raggedright\strut
		Alta\strut
	\end{minipage}\tabularnewline
	\bottomrule
	\caption{Generación de mapa.}
	\label{cu:2}
\end{longtable}
\newpage


%Caso de Uso 2.1%
\begin{longtable}[H]{@{}l|l@{}}
	\toprule
	\begin{minipage}[b]{0.23\columnwidth}\raggedright\strut
		\textbf{CU-2.1}\strut
	\end{minipage} & \begin{minipage}[b]{0.71\columnwidth}\raggedright\strut
		\textbf{Visualización de puntos sobre Google Maps}\strut
	\end{minipage}\tabularnewline
	\toprule
	\endhead
	\begin{minipage}[t]{0.23\columnwidth}\raggedright\strut
		\textbf{Requisitos asociados}\strut
	\end{minipage} & \begin{minipage}[t]{0.71\columnwidth}\raggedright\strut
		RF-2.1, RF-2.1.1\strut
	\end{minipage}\tabularnewline
	\midrule
	\begin{minipage}[t]{0.23\columnwidth}\raggedright\strut
		\textbf{Actor}\strut
	\end{minipage} & \begin{minipage}[t]{0.71\columnwidth}\raggedright\strut
		Usuario logueado.\strut
	\end{minipage}\tabularnewline
	\midrule
	\begin{minipage}[t]{0.23\columnwidth}\raggedright\strut
		\textbf{Descripción}\strut
	\end{minipage} & \begin{minipage}[t]{0.71\columnwidth}\raggedright\strut
		El usuario visualiza un mapa de Google Maps.\strut
	\end{minipage}\tabularnewline
	\midrule
	\begin{minipage}[t]{0.23\columnwidth}\raggedright\strut
		\textbf{Precondición}\strut
	\end{minipage} & \begin{minipage}[t]{0.71\columnwidth}\raggedright\strut
		\begin{itemize}
			\item Usuario logueado.
			\item API\_KEY configurada correctamente por el administrador.
			\item API de Google cargada corretamente en el navegador.
		\end{itemize}
	\end{minipage}\tabularnewline
	\midrule
	\begin{minipage}[t]{0.23\columnwidth}\raggedright\strut
		\textbf{Acciones}\strut
	\end{minipage} & \begin{minipage}[t]{0.71\columnwidth}\raggedright
		
	\end{minipage}\tabularnewline
	\midrule
	\begin{minipage}[t]{0.23\columnwidth}\raggedright\strut
		\textbf{Postcondición}\strut
	\end{minipage} & \begin{minipage}[t]{0.71\columnwidth}\raggedright\strut
		Seleccionar puntos (\nameref{cu:2.2}).\strut
	\end{minipage}\tabularnewline
	\midrule
	\begin{minipage}[t]{0.23\columnwidth}\raggedright\strut
		\textbf{Excepciones}\strut
	\end{minipage} & \begin{minipage}[t]{0.71\columnwidth}\raggedright
		\begin{itemize}
			\tightlist
			\item API bloqueada por el administrador.
			\item Dominio bloqueado por el administrador.
			\item Error al cargadar JavaScript en el navegador.
		\end{itemize}
	\end{minipage}\tabularnewline
	\midrule
	\begin{minipage}[t]{0.23\columnwidth}\raggedright\strut
		\textbf{Importancia}\strut
	\end{minipage} & \begin{minipage}[t]{0.71\columnwidth}\raggedright\strut
		Alta\strut
	\end{minipage}\tabularnewline
	\bottomrule
	\caption{Visualización de puntos sobre Google Maps.}
	\label{cu:2.1}
\end{longtable}
\newpage

%Caso de Uso 2.1.1%
\begin{longtable}[H]{@{}l|l@{}}
	\toprule
	\begin{minipage}[b]{0.23\columnwidth}\raggedright\strut
		\textbf{CU-2.1.1}\strut
	\end{minipage} & \begin{minipage}[b]{0.71\columnwidth}\raggedright\strut
		\textbf{Google API}\strut
	\end{minipage}\tabularnewline
	\toprule
	\endhead
	\begin{minipage}[t]{0.23\columnwidth}\raggedright\strut
		\textbf{Requisitos asociados}\strut
	\end{minipage} & \begin{minipage}[t]{0.71\columnwidth}\raggedright\strut
		RF-2.1.1\strut
	\end{minipage}\tabularnewline
	\midrule
	\begin{minipage}[t]{0.23\columnwidth}\raggedright\strut
		\textbf{Actor}\strut
	\end{minipage} & \begin{minipage}[t]{0.71\columnwidth}\raggedright\strut
		Usuario logueado.\strut
	\end{minipage}\tabularnewline
	\midrule
	\begin{minipage}[t]{0.23\columnwidth}\raggedright\strut
		\textbf{Descripción}\strut
	\end{minipage} & \begin{minipage}[t]{0.71\columnwidth}\raggedright\strut
		Acceso al API de Google.\strut
	\end{minipage}\tabularnewline
	\midrule
	\begin{minipage}[t]{0.23\columnwidth}\raggedright\strut
		\textbf{Precondición}\strut
	\end{minipage} & \begin{minipage}[t]{0.71\columnwidth}\raggedright\strut
		\begin{itemize}
			\item Usuario logueado.
			\item Credenciales configuradas correctamente por el administrador.
		\end{itemize}
	\end{minipage}\tabularnewline
	\midrule
	\begin{minipage}[t]{0.23\columnwidth}\raggedright\strut
		\textbf{Acciones}\strut
	\end{minipage} & \begin{minipage}[t]{0.71\columnwidth}\raggedright
		
	\end{minipage}\tabularnewline
	\midrule
	\begin{minipage}[t]{0.23\columnwidth}\raggedright\strut
		\textbf{Postcondición}\strut
	\end{minipage} & \begin{minipage}[t]{0.71\columnwidth}\raggedright\strut
		Visualización del mapa.\strut
	\end{minipage}\tabularnewline
	\midrule
	\begin{minipage}[t]{0.23\columnwidth}\raggedright\strut
		\textbf{Excepciones}\strut
	\end{minipage} & \begin{minipage}[t]{0.71\columnwidth}\raggedright
		\begin{itemize}
			\tightlist
			\item API bloqueada por el administrador.
		\end{itemize}
	\end{minipage}\tabularnewline
	\midrule
	\begin{minipage}[t]{0.23\columnwidth}\raggedright\strut
		\textbf{Importancia}\strut
	\end{minipage} & \begin{minipage}[t]{0.71\columnwidth}\raggedright\strut
		Alta\strut
	\end{minipage}\tabularnewline
	\bottomrule
	\caption{Google API.}
	\label{cu:2.1.1}
\end{longtable}
\newpage

%Caso de Uso 2.2%
\begin{longtable}[H]{@{}l|l@{}}
	\toprule
	\begin{minipage}[b]{0.23\columnwidth}\raggedright\strut
		\textbf{CU-2.2}\strut
	\end{minipage} & \begin{minipage}[b]{0.71\columnwidth}\raggedright\strut
		\textbf{Selección de puntos}\strut
	\end{minipage}\tabularnewline
	\toprule
	\endhead
	\begin{minipage}[t]{0.23\columnwidth}\raggedright\strut
		\textbf{Requisitos asociados}\strut
	\end{minipage} & \begin{minipage}[t]{0.71\columnwidth}\raggedright\strut
		RF-2.2, RF-2.2.1, RF-2.2.2\strut
	\end{minipage}\tabularnewline
	\midrule
	\begin{minipage}[t]{0.23\columnwidth}\raggedright\strut
		\textbf{Actor}\strut
	\end{minipage} & \begin{minipage}[t]{0.71\columnwidth}\raggedright\strut
		Usuario logueado.\strut
	\end{minipage}\tabularnewline
	\midrule
	\begin{minipage}[t]{0.23\columnwidth}\raggedright\strut
		\textbf{Descripción}\strut
	\end{minipage} & \begin{minipage}[t]{0.71\columnwidth}\raggedright\strut
		Selección de puntos sobre el mapa de Google.\strut
	\end{minipage}\tabularnewline
	\midrule
	\begin{minipage}[t]{0.23\columnwidth}\raggedright\strut
		\textbf{Precondición}\strut
	\end{minipage} & \begin{minipage}[t]{0.71\columnwidth}\raggedright\strut
		\begin{itemize}
			\item Usuario logueado.
			\item Mapa cargado.
			\item JavaScript cargado.
		\end{itemize}
	\end{minipage}\tabularnewline
	\midrule
	\begin{minipage}[t]{0.23\columnwidth}\raggedright\strut
		\textbf{Acciones}\strut
	\end{minipage} & \begin{minipage}[t]{0.71\columnwidth}\raggedright
			\begin{enumerate}
			\def\labelenumi{\arabic{enumi}.}
			\tightlist
			\item Hacer click sobre el punto o lugar del mapa deseado para marcarlo.
		\end{enumerate}
	\end{minipage}\tabularnewline
	\midrule
	\begin{minipage}[t]{0.23\columnwidth}\raggedright\strut
		\textbf{Postcondición}\strut
	\end{minipage} & \begin{minipage}[t]{0.71\columnwidth}\raggedright\strut
		\begin{itemize}
			\item Aparece sobre el mapa un chincheta indicando el punto seleccionado.
			\item El lugar seleccionado aparece en una tabla junto con el resto de puntos seleccionados.
		\end{itemize}
	\end{minipage}\tabularnewline
	\midrule
	\begin{minipage}[t]{0.23\columnwidth}\raggedright\strut
		\textbf{Excepciones}\strut
	\end{minipage} & \begin{minipage}[t]{0.71\columnwidth}\raggedright
		\begin{itemize}
			\tightlist
			\item API bloqueada por el administrador.
		\end{itemize}
	\end{minipage}\tabularnewline
	\midrule
	\begin{minipage}[t]{0.23\columnwidth}\raggedright\strut
		\textbf{Importancia}\strut
	\end{minipage} & \begin{minipage}[t]{0.71\columnwidth}\raggedright\strut
		Alta\strut
	\end{minipage}\tabularnewline
	\bottomrule
	\caption{Selección de puntos.}
	\label{cu:2.2}
\end{longtable}
\newpage

%Caso de Uso 2.2.1%
\begin{longtable}[H]{@{}l|l@{}}
	\toprule
	\begin{minipage}[b]{0.23\columnwidth}\raggedright\strut
		\textbf{CU-2.2.1}\strut
	\end{minipage} & \begin{minipage}[b]{0.71\columnwidth}\raggedright\strut
		\textbf{Selección como centrales}\strut
	\end{minipage}\tabularnewline
	\toprule
	\endhead
	\begin{minipage}[t]{0.23\columnwidth}\raggedright\strut
		\textbf{Requisitos asociados}\strut
	\end{minipage} & \begin{minipage}[t]{0.71\columnwidth}\raggedright\strut
		RF-2.2, RF-2.2.1\strut
	\end{minipage}\tabularnewline
	\midrule
	\begin{minipage}[t]{0.23\columnwidth}\raggedright\strut
		\textbf{Actor}\strut
	\end{minipage} & \begin{minipage}[t]{0.71\columnwidth}\raggedright\strut
		Usuario logueado.\strut
	\end{minipage}\tabularnewline
	\midrule
	\begin{minipage}[t]{0.23\columnwidth}\raggedright\strut
		\textbf{Descripción}\strut
	\end{minipage} & \begin{minipage}[t]{0.71\columnwidth}\raggedright\strut
		Selección de algunos puntos como centrales: adquieren mayor importancia en el recorrido.\strut
	\end{minipage}\tabularnewline
	\midrule
	\begin{minipage}[t]{0.23\columnwidth}\raggedright\strut
		\textbf{Precondición}\strut
	\end{minipage} & \begin{minipage}[t]{0.71\columnwidth}\raggedright\strut
		\begin{itemize}
			\item Usuario logueado.
			\item Mapa cargado.
			\item JavaScript cargado.
			\item Punto seleccionado.
		\end{itemize}
	\end{minipage}\tabularnewline
	\midrule
	\begin{minipage}[t]{0.23\columnwidth}\raggedright\strut
		\textbf{Acciones}\strut
	\end{minipage} & \begin{minipage}[t]{0.71\columnwidth}\raggedright
		\begin{enumerate}
			\def\labelenumi{\arabic{enumi}.}
			\tightlist
			\item Hacer click sobre el \textit{checkbox} correspondiente.
			\item Para eliminarlo de centrales, volver a hacer click sobre el mismo \textit{checkbox}.
		\end{enumerate}
	\end{minipage}\tabularnewline
	\midrule
	\begin{minipage}[t]{0.23\columnwidth}\raggedright\strut
		\textbf{Postcondición}\strut
	\end{minipage} & \begin{minipage}[t]{0.71\columnwidth}\raggedright\strut
		La chincheta del punto seleccionado como central cambia de roja a verde.\strut
	\end{minipage}\tabularnewline
	\midrule
	\begin{minipage}[t]{0.23\columnwidth}\raggedright\strut
		\textbf{Excepciones}\strut
	\end{minipage} & \begin{minipage}[t]{0.71\columnwidth}\raggedright
		
	\end{minipage}\tabularnewline
	\midrule
	\begin{minipage}[t]{0.23\columnwidth}\raggedright\strut
		\textbf{Importancia}\strut
	\end{minipage} & \begin{minipage}[t]{0.71\columnwidth}\raggedright\strut
		Baja\strut
	\end{minipage}\tabularnewline
	\bottomrule
	\caption{Selección como centrales.}
	\label{cu:2.2.1}
\end{longtable}
\newpage

%Caso de Uso 2.2.2%
\begin{longtable}[H]{@{}l|l@{}}
	\toprule
	\begin{minipage}[b]{0.23\columnwidth}\raggedright\strut
		\textbf{CU-2.2.2}\strut
	\end{minipage} & \begin{minipage}[b]{0.71\columnwidth}\raggedright\strut
		\textbf{Eliminación de un punto cualquiera}\strut
	\end{minipage}\tabularnewline
	\toprule
	\endhead
	\begin{minipage}[t]{0.23\columnwidth}\raggedright\strut
		\textbf{Requisitos asociados}\strut
	\end{minipage} & \begin{minipage}[t]{0.71\columnwidth}\raggedright\strut
		RF-2.2, RF-2.2.2\strut
	\end{minipage}\tabularnewline
	\midrule
	\begin{minipage}[t]{0.23\columnwidth}\raggedright\strut
		\textbf{Actor}\strut
	\end{minipage} & \begin{minipage}[t]{0.71\columnwidth}\raggedright\strut
		Usuario logueado.\strut
	\end{minipage}\tabularnewline
	\midrule
	\begin{minipage}[t]{0.23\columnwidth}\raggedright\strut
		\textbf{Descripción}\strut
	\end{minipage} & \begin{minipage}[t]{0.71\columnwidth}\raggedright\strut
		Eliminar un punto seleccionado previamente.\strut
	\end{minipage}\tabularnewline
	\midrule
	\begin{minipage}[t]{0.23\columnwidth}\raggedright\strut
		\textbf{Precondición}\strut
	\end{minipage} & \begin{minipage}[t]{0.71\columnwidth}\raggedright\strut
		\begin{itemize}
			\item Usuario logueado.
			\item Mapa cargado.
			\item JavaScript cargado.
			\item Punto seleccionado.
			\item Punto añadido a la tabla que contiene la lista de los lugares ya seleccionados.
		\end{itemize}
	\end{minipage}\tabularnewline
	\midrule
	\begin{minipage}[t]{0.23\columnwidth}\raggedright\strut
		\textbf{Acciones}\strut
	\end{minipage} & \begin{minipage}[t]{0.71\columnwidth}\raggedright
		\begin{enumerate}
			\def\labelenumi{\arabic{enumi}.}
			\tightlist
			\item Hacer click sobre el icono de la papelera del punto que se quiere eliminar.
		\end{enumerate}
	\end{minipage}\tabularnewline
	\midrule
	\begin{minipage}[t]{0.23\columnwidth}\raggedright\strut
		\textbf{Postcondición}\strut
	\end{minipage} & \begin{minipage}[t]{0.71\columnwidth}\raggedright\strut
		El punto desaparece del mapa y de la tabla.\strut
	\end{minipage}\tabularnewline
	\midrule
	\begin{minipage}[t]{0.23\columnwidth}\raggedright\strut
		\textbf{Excepciones}\strut
	\end{minipage} & \begin{minipage}[t]{0.71\columnwidth}\raggedright
		
	\end{minipage}\tabularnewline
	\midrule
	\begin{minipage}[t]{0.23\columnwidth}\raggedright\strut
		\textbf{Importancia}\strut
	\end{minipage} & \begin{minipage}[t]{0.71\columnwidth}\raggedright\strut
		Baja\strut
	\end{minipage}\tabularnewline
	\bottomrule
	\caption{Eliminación de un punto cualquiera.}
	\label{cu:2.2.2}
\end{longtable}
\newpage

%Caso de Uso 3%
\begin{longtable}[H]{@{}l|l@{}}
	\toprule
	\begin{minipage}[b]{0.23\columnwidth}\raggedright\strut
		\textbf{CU-3}\strut
	\end{minipage} & \begin{minipage}[b]{0.71\columnwidth}\raggedright\strut
		\textbf{Creación de SVG}\strut
	\end{minipage}\tabularnewline
	\toprule
	\endhead
	\begin{minipage}[t]{0.23\columnwidth}\raggedright\strut
		\textbf{Requisitos asociados}\strut
	\end{minipage} & \begin{minipage}[t]{0.71\columnwidth}\raggedright\strut
		RF-3, RF-3.1, RF-3.2, RF-3.3, RF-3.4\strut
	\end{minipage}\tabularnewline
	\midrule
	\begin{minipage}[t]{0.23\columnwidth}\raggedright\strut
		\textbf{Actor}\strut
	\end{minipage} & \begin{minipage}[t]{0.71\columnwidth}\raggedright\strut
		Usuario logueado.\strut
	\end{minipage}\tabularnewline
	\midrule
	\begin{minipage}[t]{0.23\columnwidth}\raggedright\strut
		\textbf{Descripción}\strut
	\end{minipage} & \begin{minipage}[t]{0.71\columnwidth}\raggedright\strut
		El usuario crea un SVG que puede editar.\strut
	\end{minipage}\tabularnewline
	\midrule
	\begin{minipage}[t]{0.23\columnwidth}\raggedright\strut
		\textbf{Precondición}\strut
	\end{minipage} & \begin{minipage}[t]{0.71\columnwidth}\raggedright\strut
		Sesión iniciada, puntos seleccionados.\strut
	\end{minipage}\tabularnewline
	\midrule
	\begin{minipage}[t]{0.23\columnwidth}\raggedright\strut
		\textbf{Acciones}\strut
	\end{minipage} & \begin{minipage}[t]{0.71\columnwidth}\raggedright
		\begin{enumerate}
			\def\labelenumi{\arabic{enumi}.}
			\tightlist
			\item Seleccionar el mapa más conveniente.
			\item Enviar y calcular/generar el mapa.
			\item Editar las posiciones de los puntos.
			\item Editar los nombres de los puntos.
			\item Editar la posición de los textos.
			\item Exportar el mapa.
		\end{enumerate}
	\end{minipage}\tabularnewline
	\midrule
	\begin{minipage}[t]{0.23\columnwidth}\raggedright\strut
		\textbf{Postcondición}\strut
	\end{minipage} & \begin{minipage}[t]{0.71\columnwidth}\raggedright\strut
		Nuevo cálculo de las posiciones de los textos al mover un punto.\strut
	\end{minipage}\tabularnewline
	\midrule
	\begin{minipage}[t]{0.23\columnwidth}\raggedright\strut
		\textbf{Excepciones}\strut
	\end{minipage} & \begin{minipage}[t]{0.71\columnwidth}\raggedright

	\end{minipage}\tabularnewline
	\midrule
	\begin{minipage}[t]{0.23\columnwidth}\raggedright\strut
		\textbf{Importancia}\strut
	\end{minipage} & \begin{minipage}[t]{0.71\columnwidth}\raggedright\strut
		Alta\strut
	\end{minipage}\tabularnewline
	\bottomrule
	\caption{Creación de SVG.}
	\label{cu:3}
\end{longtable}
\newpage


%Caso de Uso 3.1%
\begin{longtable}[H]{@{}l|l@{}}
	\toprule
	\begin{minipage}[b]{0.23\columnwidth}\raggedright\strut
		\textbf{CU-3.1}\strut
	\end{minipage} & \begin{minipage}[b]{0.71\columnwidth}\raggedright\strut
		\textbf{Añadir y eliminar arcos}\strut
	\end{minipage}\tabularnewline
	\toprule
	\endhead
	\begin{minipage}[t]{0.23\columnwidth}\raggedright\strut
		\textbf{Requisitos asociados}\strut
	\end{minipage} & \begin{minipage}[t]{0.71\columnwidth}\raggedright\strut
		RF-3.1\strut
	\end{minipage}\tabularnewline
	\midrule
	\begin{minipage}[t]{0.23\columnwidth}\raggedright\strut
		\textbf{Actor}\strut
	\end{minipage} & \begin{minipage}[t]{0.71\columnwidth}\raggedright\strut
		Usuario logueado.\strut
	\end{minipage}\tabularnewline
	\midrule
	\begin{minipage}[t]{0.23\columnwidth}\raggedright\strut
		\textbf{Descripción}\strut
	\end{minipage} & \begin{minipage}[t]{0.71\columnwidth}\raggedright\strut
		Mediante el uso de un slider, el usuario elige el número de arcos (trayectos) que se muestran.\strut
	\end{minipage}\tabularnewline
	\midrule
	\begin{minipage}[t]{0.23\columnwidth}\raggedright\strut
		\textbf{Precondición}\strut
	\end{minipage} & \begin{minipage}[t]{0.71\columnwidth}\raggedright
		\begin{itemize}
			\item Sesión iniciada.
			\item Puntos seleccionados y enviados (\nameref{cu:2}).
		\end{itemize}.
	\end{minipage}\tabularnewline
	\midrule
	\begin{minipage}[t]{0.23\columnwidth}\raggedright\strut
		\textbf{Acciones}\strut
	\end{minipage} & \begin{minipage}[t]{0.71\columnwidth}\raggedright
		\begin{enumerate}
			\def\labelenumi{\arabic{enumi}.}
			\tightlist
			\item Mover el punto del slider.
			\item Enviar el mapa seleccionado. 
		\end{enumerate}
	\end{minipage}\tabularnewline
	\midrule
	\begin{minipage}[t]{0.23\columnwidth}\raggedright\strut
		\textbf{Postcondición}\strut
	\end{minipage} & \begin{minipage}[t]{0.71\columnwidth}\raggedright\strut
		Redirección a una nueva página en la que solo se muestra el mapa seleccionado.\strut
	\end{minipage}\tabularnewline
	\midrule
	\begin{minipage}[t]{0.23\columnwidth}\raggedright\strut
		\textbf{Excepciones}\strut
	\end{minipage} & \begin{minipage}[t]{0.71\columnwidth}\raggedright
		
	\end{minipage}\tabularnewline
	\midrule
	\begin{minipage}[t]{0.23\columnwidth}\raggedright\strut
		\textbf{Importancia}\strut
	\end{minipage} & \begin{minipage}[t]{0.71\columnwidth}\raggedright\strut
		Alta\strut
	\end{minipage}\tabularnewline
	\bottomrule
	\caption{Añadir y eliminar arcos.}
	\label{cu:3.1}
\end{longtable}
\newpage

%Caso de Uso 3.2%
\begin{longtable}[H]{@{}l|l@{}}
	\toprule
	\begin{minipage}[b]{0.23\columnwidth}\raggedright\strut
		\textbf{CU-3.2}\strut
	\end{minipage} & \begin{minipage}[b]{0.71\columnwidth}\raggedright\strut
		\textbf{Mover los puntos y las etiquetas}\strut
	\end{minipage}\tabularnewline
	\toprule
	\endhead
	\begin{minipage}[t]{0.23\columnwidth}\raggedright\strut
		\textbf{Requisitos asociados}\strut
	\end{minipage} & \begin{minipage}[t]{0.71\columnwidth}\raggedright\strut
		RF-3.2\strut
	\end{minipage}\tabularnewline
	\midrule
	\begin{minipage}[t]{0.23\columnwidth}\raggedright\strut
		\textbf{Actor}\strut
	\end{minipage} & \begin{minipage}[t]{0.71\columnwidth}\raggedright\strut
		Usuario logueado.\strut
	\end{minipage}\tabularnewline
	\midrule
	\begin{minipage}[t]{0.23\columnwidth}\raggedright\strut
		\textbf{Descripción}\strut
	\end{minipage} & \begin{minipage}[t]{0.71\columnwidth}\raggedright\strut
		Cambiar de posición los puntos y etiquetas del mapa.\strut
	\end{minipage}\tabularnewline
	\midrule
	\begin{minipage}[t]{0.23\columnwidth}\raggedright\strut
		\textbf{Precondición}\strut
	\end{minipage} & \begin{minipage}[t]{0.71\columnwidth}\raggedright
		\begin{itemize}
			\item Sesión iniciada.
			\item Mapa cargado.
		\end{itemize}.
	\end{minipage}\tabularnewline
	\midrule
	\begin{minipage}[t]{0.23\columnwidth}\raggedright\strut
		\textbf{Acciones}\strut
	\end{minipage} & \begin{minipage}[t]{0.71\columnwidth}\raggedright
		\begin{enumerate}
			\def\labelenumi{\arabic{enumi}.}
			\tightlist
			\item Mover el punto.
			\item Mover las etiquetas. 
		\end{enumerate}
	\end{minipage}\tabularnewline
	\midrule
	\begin{minipage}[t]{0.23\columnwidth}\raggedright\strut
		\textbf{Postcondición}\strut
	\end{minipage} & \begin{minipage}[t]{0.71\columnwidth}\raggedright\strut
		\begin{itemize}
			\item Cambio de posición.
			\item La posición de las etiquetas se recalcula para ajustarse a la nueva posición del punto.
		\end{itemize}
	\end{minipage}\tabularnewline
	\midrule
	\begin{minipage}[t]{0.23\columnwidth}\raggedright\strut
		\textbf{Excepciones}\strut
	\end{minipage} & \begin{minipage}[t]{0.71\columnwidth}\raggedright
		
	\end{minipage}\tabularnewline
	\midrule
	\begin{minipage}[t]{0.23\columnwidth}\raggedright\strut
		\textbf{Importancia}\strut
	\end{minipage} & \begin{minipage}[t]{0.71\columnwidth}\raggedright\strut
		Baja\strut
	\end{minipage}\tabularnewline
	\bottomrule
	\caption{Mover los puntos y las etiquetas.}
	\label{cu:3.2}
\end{longtable}
\newpage

%Caso de Uso 3.3%
\begin{longtable}[H]{@{}l|l@{}}
	\toprule
	\begin{minipage}[b]{0.23\columnwidth}\raggedright\strut
		\textbf{CU-3.3}\strut
	\end{minipage} & \begin{minipage}[b]{0.71\columnwidth}\raggedright\strut
		\textbf{Cambiar el nombre de los puntos}\strut
	\end{minipage}\tabularnewline
	\toprule
	\endhead
	\begin{minipage}[t]{0.23\columnwidth}\raggedright\strut
		\textbf{Requisitos asociados}\strut
	\end{minipage} & \begin{minipage}[t]{0.71\columnwidth}\raggedright\strut
		RF-3.3\strut
	\end{minipage}\tabularnewline
	\midrule
	\begin{minipage}[t]{0.23\columnwidth}\raggedright\strut
		\textbf{Actor}\strut
	\end{minipage} & \begin{minipage}[t]{0.71\columnwidth}\raggedright\strut
		Usuario logueado.\strut
	\end{minipage}\tabularnewline
	\midrule
	\begin{minipage}[t]{0.23\columnwidth}\raggedright\strut
		\textbf{Descripción}\strut
	\end{minipage} & \begin{minipage}[t]{0.71\columnwidth}\raggedright\strut
		Cambiar el nombre de los puntos que aparecen en el mapa.\strut
	\end{minipage}\tabularnewline
	\midrule
	\begin{minipage}[t]{0.23\columnwidth}\raggedright\strut
		\textbf{Precondición}\strut
	\end{minipage} & \begin{minipage}[t]{0.71\columnwidth}\raggedright
		\begin{itemize}
			\item Sesión iniciada.
			\item Mapa cargado.
		\end{itemize}.
	\end{minipage}\tabularnewline
	\midrule
	\begin{minipage}[t]{0.23\columnwidth}\raggedright\strut
		\textbf{Acciones}\strut
	\end{minipage} & \begin{minipage}[t]{0.71\columnwidth}\raggedright
		\begin{enumerate}
			\def\labelenumi{\arabic{enumi}.}
			\tightlist
			\item Pulsar sobre la etiqueta que se quiere cambiar.
			\item Introducir el texto nuevo. 
		\end{enumerate}
	\end{minipage}\tabularnewline
	\midrule
	\begin{minipage}[t]{0.23\columnwidth}\raggedright\strut
		\textbf{Postcondición}\strut
	\end{minipage} & \begin{minipage}[t]{0.71\columnwidth}\raggedright\strut
		Cambio de texto de la etiqueta.
	\end{minipage}\tabularnewline
	\midrule
	\begin{minipage}[t]{0.23\columnwidth}\raggedright\strut
		\textbf{Excepciones}\strut
	\end{minipage} & \begin{minipage}[t]{0.71\columnwidth}\raggedright
		
	\end{minipage}\tabularnewline
	\midrule
	\begin{minipage}[t]{0.23\columnwidth}\raggedright\strut
		\textbf{Importancia}\strut
	\end{minipage} & \begin{minipage}[t]{0.71\columnwidth}\raggedright\strut
		Baja\strut
	\end{minipage}\tabularnewline
	\bottomrule
	\caption{Cambiar el nombre de los puntos.}
	\label{cu:3.3}
\end{longtable}
\newpage

%Caso de Uso 3.4%
\begin{longtable}[H]{@{}l|l@{}}
	\toprule
	\begin{minipage}[b]{0.23\columnwidth}\raggedright\strut
		\textbf{CU-3.4}\strut
	\end{minipage} & \begin{minipage}[b]{0.71\columnwidth}\raggedright\strut
		\textbf{Exportación y descarga}\strut
	\end{minipage}\tabularnewline
	\toprule
	\endhead
	\begin{minipage}[t]{0.23\columnwidth}\raggedright\strut
		\textbf{Requisitos asociados}\strut
	\end{minipage} & \begin{minipage}[t]{0.71\columnwidth}\raggedright\strut
		RF-3.4\strut
	\end{minipage}\tabularnewline
	\midrule
	\begin{minipage}[t]{0.23\columnwidth}\raggedright\strut
		\textbf{Actor}\strut
	\end{minipage} & \begin{minipage}[t]{0.71\columnwidth}\raggedright\strut
		Usuario logueado.\strut
	\end{minipage}\tabularnewline
	\midrule
	\begin{minipage}[t]{0.23\columnwidth}\raggedright\strut
		\textbf{Descripción}\strut
	\end{minipage} & \begin{minipage}[t]{0.71\columnwidth}\raggedright\strut
		Descarga como SVG del mapa generado (editado o no).\strut
	\end{minipage}\tabularnewline
	\midrule
	\begin{minipage}[t]{0.23\columnwidth}\raggedright\strut
		\textbf{Precondición}\strut
	\end{minipage} & \begin{minipage}[t]{0.71\columnwidth}\raggedright
		\begin{itemize}
			\item Sesión iniciada.
			\item Mapa cargado.
		\end{itemize}.
	\end{minipage}\tabularnewline
	\midrule
	\begin{minipage}[t]{0.23\columnwidth}\raggedright\strut
		\textbf{Acciones}\strut
	\end{minipage} & \begin{minipage}[t]{0.71\columnwidth}\raggedright
		\begin{enumerate}
			\def\labelenumi{\arabic{enumi}.}
			\tightlist
			\item Pulsar sobre el botón de exportar.
			\item Guardar archivo. 
		\end{enumerate}
	\end{minipage}\tabularnewline
	\midrule
	\begin{minipage}[t]{0.23\columnwidth}\raggedright\strut
		\textbf{Postcondición}\strut
	\end{minipage} & \begin{minipage}[t]{0.71\columnwidth}\raggedright\strut
		Descarga del mapa.
	\end{minipage}\tabularnewline
	\midrule
	\begin{minipage}[t]{0.23\columnwidth}\raggedright\strut
		\textbf{Excepciones}\strut
	\end{minipage} & \begin{minipage}[t]{0.71\columnwidth}\raggedright
		
	\end{minipage}\tabularnewline
	\midrule
	\begin{minipage}[t]{0.23\columnwidth}\raggedright\strut
		\textbf{Importancia}\strut
	\end{minipage} & \begin{minipage}[t]{0.71\columnwidth}\raggedright\strut
		Baja\strut
	\end{minipage}\tabularnewline
	\bottomrule
	\caption{Exportación y descarga.}
	\label{cu:3.4}
\end{longtable}
\newpage

