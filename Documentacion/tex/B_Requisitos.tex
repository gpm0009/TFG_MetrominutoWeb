\apendice{Especificación de Requisitos}

\section{Introducción}
En este apéndice se explican y especifican tanto los requisitos funcionales como los no funcionales del proyecto, así como los objetivos del proyecto.

\section{Objetivos generales}

\begin{itemize}
	\item Crear una aplicación cliente -- servidor que permita la creación automática de Metrominutos.
    \item Control de usuarios.
    \item Permitir a los usuarios control sobre el mapa, de manera que puedan mover o eliminar los puntos seleccionados.
    \item Ofrecer al usuario un mapa final claro y sencillo.
    \item Que la aplicación final sea útil para el fomento de esta actividad.
\end{itemize}

\section{Catálogo de requisitos}

\subsection{Requisitos funcionales}
\begin{itemize}
	\item \textbf{RF-1 Control de usuarios}: la aplicación debe permitir el control de usuarios.
	\begin{itemize}
		\item \textbf{RF-1.1 Firebase Auth:} La aplicación debe poder hacer uso de los servicios de autenticación de Firebase.
	\end{itemize}
	\item \textbf{RF-2 Generación de mapa:} La aplicación debe ofrecer la selección de distintos puntos sobre el mapa para generar un mapa personalizado.
	\begin{itemize}
		\item \textbf{RF-2.1 Visualización de puntos sobre Google Maps:} La aplicación debe poder hacer uso de las operaciones, así como acceder a los servicios de \textit{Google Maps} proporcionados por el API.
		\begin{itemize}
			\item \textbf{RF-2.1.1 Google Api:}
		\end{itemize}
		\item \textbf{RF-2.2 Selección de puntos:} Seleccionar sobre el mapa distintos puntos, hasta un máximo de 15.
		\begin{itemize}
			\item \textbf{RF-2.2.1 Selección como centrales:} Una vez creado un marcador, el usuario debe poder darle una mayor importancia a este punto en el recorrido.
			\item \textbf{RF-2.2.2 Eliminación de un punto cualquiera:} El usuario debe poder eliminar los puntos seleccionados. 
		\end{itemize}
		
	\end{itemize}
	\item \textbf{RF-3 Creación de SVG:} Generación de un mapa sinóptico a través de los puntos seleccionados en el \textit{RF-~2}, con formato SVG, que permite la interacción con dicho mapa.
	\begin{itemize}
		\item \textbf{RF-3.1 Añadir y eliminar arcos:} El usuario debe poder seleccionar el mapa o grafo que más se adecue a su necesidad, con más o menos recorridos (arcos).
		\item \textbf{RF-3.2 Mover puntos y etiquetas:} Posibilidad de que el usuario recoloque tanto los puntos como las etiquetas de los mismos en el mapa.
		\item \textbf{RF-3.3 Cambiar el nombre de los puntos:} El usuario debe poder cambiar las etiquetas referentes al nombre de los puntos.
		\item \textbf{RF-3.4 Exportación y descarga:} El usuario debe poder descargar el mapa que ha generado.
	\end{itemize}
\end{itemize}


\subsection{Requisitos no funcionales}
\begin{itemize}
	\item \textbf{RNF-1 Usabilidad:} La aplicación tiene que poder usarse de forma sencilla y debe ser intuitiva.
	\item \textbf{RNF-2 Compatibilidad:} La aplicación tiene que poder ser compatible con los diferentes navegadores.
	\item \textbf{RNF-3 Responsividad:} La aplicación debe poder adaptarse al tamaño de la pantalla.
	\item \textbf{RNF-4 Facilidad en el despliegue:} La aplicación debe poder ser desplegada con facilidad en un servidor.
	\item \textbf{RFN-5 Mantenibilidad:} Debe ser sencillo añadir nuevas funcionalidades.
\end{itemize}

\section{Especificación de requisitos}
En esta sección, se presentan los casos de uso de la aplicación.
\imagen{Diagrama de caso de uso - sitema}{Diagrama de casos de uso: General.}
\imagen{Diagrama de caso de uso - RF2}{Diagrama de casos de uso: RF2 -- Generación de mapa.}
\imagen{Diagrama de caso de uso - RF3}{Diagrama de casos de uso: RF3 -- Creación de SVG.}



\begin{longtable}[H]{@{}l|l@{}}
	\toprule
	\begin{minipage}[b]{0.23\columnwidth}\raggedright\strut
		\textbf{CU-1}\strut
	\end{minipage} & \begin{minipage}[b]{0.71\columnwidth}\raggedright\strut
		\textbf{Iniciar sesión}\strut
	\end{minipage}\tabularnewline
	\toprule
	\endhead
	\begin{minipage}[t]{0.23\columnwidth}\raggedright\strut
		\textbf{Requisitos asociados}\strut
	\end{minipage} & \begin{minipage}[t]{0.71\columnwidth}\raggedright\strut
		RF-1, RF-1.1\strut
	\end{minipage}\tabularnewline
	\midrule
	\begin{minipage}[t]{0.23\columnwidth}\raggedright\strut
		\textbf{Actor}\strut
	\end{minipage} & \begin{minipage}[t]{0.71\columnwidth}\raggedright\strut
		Usuario no logueado.\strut
	\end{minipage}\tabularnewline
	\midrule
	\begin{minipage}[t]{0.23\columnwidth}\raggedright\strut
		\textbf{Descripción}\strut
	\end{minipage} & \begin{minipage}[t]{0.71\columnwidth}\raggedright\strut
		El usuario inicia sesión en la aplicación.\strut
	\end{minipage}\tabularnewline
	\midrule
	\begin{minipage}[t]{0.23\columnwidth}\raggedright\strut
		\textbf{Precondición}\strut
	\end{minipage} & \begin{minipage}[t]{0.71\columnwidth}\raggedright\strut
		Navegador web abierto y página de inicio de la aplicación cargada.\strut
	\end{minipage}\tabularnewline
	\midrule
	\begin{minipage}[t]{0.23\columnwidth}\raggedright\strut
		\textbf{Acciones}\strut
	\end{minipage} & \begin{minipage}[t]{0.71\columnwidth}\raggedright\strut
		\begin{enumerate}
			\def\labelenumi{\arabic{enumi}.}
			\tightlist
			\item Pulsar sobre el botón Iniciar sesión.
			\item Seleccionar proveedor de autenticación.
			\item Escribir los credenciales correspondientes.
		\end{enumerate}
	\end{minipage}\tabularnewline
	\midrule
	\begin{minipage}[t]{0.23\columnwidth}\raggedright\strut
		\textbf{Postcondición}\strut
	\end{minipage} & \begin{minipage}[t]{0.71\columnwidth}\raggedright\strut
		Redirección a la aplicación con la sesión iniciada.\strut
	\end{minipage}\tabularnewline
	\midrule
	\begin{minipage}[t]{0.23\columnwidth}\raggedright\strut
		\textbf{Excepciones}\strut
	\end{minipage} & \begin{minipage}[t]{0.71\columnwidth}\raggedright\strut
		\begin{itemize}
			\tightlist
			\item Cuenta no existente.
			\item Combinación de nombre y contraseña incorrecta.
			\item Acceso no permitido.
		\end{itemize}
	\end{minipage}\tabularnewline
	\midrule
	\begin{minipage}[t]{0.23\columnwidth}\raggedright\strut
		\textbf{Importancia}\strut
	\end{minipage} & \begin{minipage}[t]{0.71\columnwidth}\raggedright\strut
		Alta\strut
	\end{minipage}\tabularnewline
	\bottomrule
	\caption{Iniciar sesión.}
	\label{cu:1}
\end{longtable}
\newpage


\begin{longtable}[H]{@{}l|l@{}}
	\toprule
	\begin{minipage}[b]{0.23\columnwidth}\raggedright\strut
		\textbf{CU-2}\strut
	\end{minipage} & \begin{minipage}[b]{0.71\columnwidth}\raggedright\strut
		\textbf{Generación de mapa}\strut
	\end{minipage}\tabularnewline
	\toprule
	\endhead
	\begin{minipage}[t]{0.23\columnwidth}\raggedright\strut
		\textbf{Requisitos asociados}\strut
	\end{minipage} & \begin{minipage}[t]{0.71\columnwidth}\raggedright\strut
		RF-2, RF-2.1, RF-2.2, RF-2.1.1, RF-2.2.1, RF-2.2.2\strut
	\end{minipage}\tabularnewline
	\midrule
	\begin{minipage}[t]{0.23\columnwidth}\raggedright\strut
		\textbf{Actor}\strut
	\end{minipage} & \begin{minipage}[t]{0.71\columnwidth}\raggedright\strut
		Usuario logueado.\strut
	\end{minipage}\tabularnewline
	\midrule
	\begin{minipage}[t]{0.23\columnwidth}\raggedright\strut
		\textbf{Descripción}\strut
	\end{minipage} & \begin{minipage}[t]{0.71\columnwidth}\raggedright\strut
		El usuario selecciona distintos puntos sobre Google Maps y genera un mapa sinóptico.\strut
	\end{minipage}\tabularnewline
	\midrule
	\begin{minipage}[t]{0.23\columnwidth}\raggedright\strut
		\textbf{Precondición}\strut
	\end{minipage} & \begin{minipage}[t]{0.71\columnwidth}\raggedright\strut
		Sesión iniciada.\strut
	\end{minipage}\tabularnewline
	\midrule
	\begin{minipage}[t]{0.23\columnwidth}\raggedright\strut
		\textbf{Acciones}\strut
	\end{minipage} & \begin{minipage}[t]{0.71\columnwidth}\raggedright\strut
		\begin{enumerate}
			\def\labelenumi{\arabic{enumi}.}
			\tightlist
			\item Pulsar sobre el mapa para marcar un punto.
			\item Pulsar sobre el icono de la papelera para borrar un punto.
			\item Pulsar sobre Borrar Marcadores para eliminar todos los puntos.
			\item Pulsar sobre Generar mapa para crear un mapa sinóptico de los puntos seleccionados.
		\end{enumerate}
	\end{minipage}\tabularnewline
	\midrule
	\begin{minipage}[t]{0.23\columnwidth}\raggedright\strut
		\textbf{Postcondición}\strut
	\end{minipage} & \begin{minipage}[t]{0.71\columnwidth}\raggedright\strut
		Redirección a la vista del mapa sinóptico.\strut
	\end{minipage}\tabularnewline
	\midrule
	\begin{minipage}[t]{0.23\columnwidth}\raggedright\strut
		\textbf{Excepciones}\strut
	\end{minipage} & \begin{minipage}[t]{0.71\columnwidth}\raggedright\strut
		\begin{itemize}
			\tightlist
			\item Seleccionar más de 15 puntos.
			\item Seleccionar menos de 3 puntos.
		\end{itemize}
	\end{minipage}\tabularnewline
	\midrule
	\begin{minipage}[t]{0.23\columnwidth}\raggedright\strut
		\textbf{Importancia}\strut
	\end{minipage} & \begin{minipage}[t]{0.71\columnwidth}\raggedright\strut
		Alta\strut
	\end{minipage}\tabularnewline
	\bottomrule
	\caption{Generación de mapa.}
	\label{cu:2}
\end{longtable}
\newpage


\begin{longtable}[H]{@{}l|l@{}}
	\toprule
	\begin{minipage}[b]{0.23\columnwidth}\raggedright\strut
		\textbf{CU-3}\strut
	\end{minipage} & \begin{minipage}[b]{0.71\columnwidth}\raggedright\strut
		\textbf{Creación de SVG}\strut
	\end{minipage}\tabularnewline
	\toprule
	\endhead
	\begin{minipage}[t]{0.23\columnwidth}\raggedright\strut
		\textbf{Requisitos asociados}\strut
	\end{minipage} & \begin{minipage}[t]{0.71\columnwidth}\raggedright\strut
		RF-3, RF-3.1, RF-3.2, RF-3.3, RF-3.4\strut
	\end{minipage}\tabularnewline
	\midrule
	\begin{minipage}[t]{0.23\columnwidth}\raggedright\strut
		\textbf{Actor}\strut
	\end{minipage} & \begin{minipage}[t]{0.71\columnwidth}\raggedright\strut
		Usuario logueado.\strut
	\end{minipage}\tabularnewline
	\midrule
	\begin{minipage}[t]{0.23\columnwidth}\raggedright\strut
		\textbf{Descripción}\strut
	\end{minipage} & \begin{minipage}[t]{0.71\columnwidth}\raggedright\strut
		El usuario crea un SVG que puede editar.\strut
	\end{minipage}\tabularnewline
	\midrule
	\begin{minipage}[t]{0.23\columnwidth}\raggedright\strut
		\textbf{Precondición}\strut
	\end{minipage} & \begin{minipage}[t]{0.71\columnwidth}\raggedright\strut
		Sesión iniciada, puntos seleccionados.\strut
	\end{minipage}\tabularnewline
	\midrule
	\begin{minipage}[t]{0.23\columnwidth}\raggedright\strut
		\textbf{Acciones}\strut
	\end{minipage} & \begin{minipage}[t]{0.71\columnwidth}\raggedright\strut
		\begin{enumerate}
			\def\labelenumi{\arabic{enumi}.}
			\tightlist
			\item Seleccionar el mapa más conveniente.
			\item Pulsar sobre Enviar.
			\item Editar las posiciones de los puntos.
			\item Editar los nombres de los puntos.
			\item Editar la posición de los textos.
			\item Exportar el mapa.
		\end{enumerate}
	\end{minipage}\tabularnewline
	\midrule
	\begin{minipage}[t]{0.23\columnwidth}\raggedright\strut
		\textbf{Postcondición}\strut
	\end{minipage} & \begin{minipage}[t]{0.71\columnwidth}\raggedright\strut
		Nuevo cálculo de las posiciones de los textos al mover un punto.\strut
	\end{minipage}\tabularnewline
	\midrule
	\begin{minipage}[t]{0.23\columnwidth}\raggedright\strut
		\textbf{Excepciones}\strut
	\end{minipage} & \begin{minipage}[t]{0.71\columnwidth}\raggedright\strut
		\begin{itemize}
			\tightlist
			\item No se puede exportar el mapa.
		\end{itemize}
	\end{minipage}\tabularnewline
	\midrule
	\begin{minipage}[t]{0.23\columnwidth}\raggedright\strut
		\textbf{Importancia}\strut
	\end{minipage} & \begin{minipage}[t]{0.71\columnwidth}\raggedright\strut
		Alta\strut
	\end{minipage}\tabularnewline
	\bottomrule
	\caption{Creación de SVG.}
	\label{cu:3}
\end{longtable}
\newpage


FIXME: He estado mirando varios ejemplos, y aqui tendría que meter otra fila en la tabla que fuese casos de extensión o de inclusión?
O añado directamente esos casos como una tabla nueva? por el momento lo he dejado incluido en la sección de pasos como un elemento de la lista (probablemente no sea la mejor opción).
