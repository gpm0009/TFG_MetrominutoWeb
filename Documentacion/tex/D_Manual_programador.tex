\apendice{Documentación técnica de programación}

\section{Introducción}
En este apéndice se va a definir todo aquello que es necesario conocer para que se pueda continuar con el desarrollo del proyecto, desde su estructura hasta una breve descripción de como instalar la aplicación y configurar nuestro entorno de trabajo para llevar a cabo el desarrollo.

\section{Estructura de directorios}
La estructura del proyecto se divide en:
\dirtree{%
	.1 / Directorio raíz.
	.2 HolaMundo/ -App web básica.
	.2 Metrominuto/ -Aplicación web.
	.3 static/ -ficheros JavaScript.
	.3 template/ -ficheros HTML.
}

\section{Manual del programador}
En este apartado se explican los puntos a tener en cuenta por futuros desarrolladores que tengan la intención de mantener o mejorar el proyecto.

\subsection{Aplicaciones utilizadas}
Para el desarrollo de este proyecto se tuvieron en cuenta principalmente dos editores de texto y dos herramientas para mantener el control de versiones.

\begin{itemize}
	\item \textbf{Visual Studio Code} 
	\item \textbf{PyCharm}
	\item \textbf{GitHub}
	\item \textbf{GitCraken}
\end{itemize}
Después de analizar y configurar ambos editores de texto, se llego a la conclusión de que era mucho mas cómodo y útil utilizar PyCharm, ya que ofrece una configuración mas sencilla, además de permitir importar diversas librerías de una manera mas amigable. También ofrece la posibilidad de seguir los estándares de programación(PEP8).

\subsection{Instalación y configuración}
Para la instalación del proyecto se explicarán los pasos a seguir en un sistema operativo de linux: en este caso he usado LINUX MINT.

\subsection{Ejecución}

\section{Compilación, instalación y ejecución del proyecto}


\section{Pruebas del sistema}
